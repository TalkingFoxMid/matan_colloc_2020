\documentclass{article}
\usepackage[utf8]{inputenc}
\usepackage[usenames]{color}
\usepackage[T2A]{fontenc}
\usepackage[russian]{babel}
\usepackage{hyphenat}
\hyphenation{ма-те-ма-ти-ка вос-ста-нав-ли-вать}
\usepackage{mathtools}
\usepackage{mathrsfs}
\usepackage{amsmath}
\usepackage{amsfonts}
\usepackage[14pt]{extsizes}
\linespread{1.15}
\pagestyle{empty}
\begin{document}
\begin{center}
\textbf{1. Понятия разбиения отрезка, диаметра разбиения, измельчения разбиения, объединения разбиений.
Понятие интегральной суммы и предела интегральных сумм. Определение определённого интеграла Римана на отрезке и функции, интегрируемой по Риману на отрезке. Геометрический смысл
определённого интеграла.}
\end{center}

\textcolor{red}{Разбиение отрезка }
[a,b] - это конечное, упорядоченное по возрастанию множество точек $T = \{a = x_0, x_1, ... , x_n=b\} : \forall i = 0, 1,..., n-1 $   $[x_i, x_{i+1}]$ .
 
 $\tau  = \{x_i\}_{i=0}^n$ - разбиение отрезка [a, b].
 
Частичный отрезок разбиения - $[x_i, x_{i+1}]$.

$\triangle x_i = x_{i+1} - x_i$ 

\textcolor{red}{Диаметр (мелкость) разбиения} $d_\tau(\lambda)  = max \triangle x_i$
$\forall i = 0, 1,..., n-1$

Измельчение разбиения  - такое разбиение, содержащее все точки разбиения и может быть каки-то дополнительных. 

$\tau \subseteq \tau' $ $d\tau' \leq d\tau $

Объединение размельчений ...

\textcolor{red}{Интегральная сумма} - $\forall \xi_i  \in  [x_i, x_{i+1}] $ $\sum_{i=0}^{n-1}f(\xi_i)\triangle x_i$ = $\sigma(\tau_i, \{\xi_i\}_{i=0}^{n-1})$, где $\tau_i$ - способ разбиения, $\{\xi_i\}_{i=0}^{n-1}$ - выбор точек.

Если существует конечный предел интегральных сумм при стремлении диаметра d разбиения к нулю, не зависящий от способа разбиения и выбора точек $\xi$, то такой предел называется \textcolor{red}{интегралом Римана} на отрезке [a, b].

$\forall \epsilon>0$  $\exists\delta(\epsilon) > 0$: $\forall\tau$ $d < \delta(\epsilon)$ $\forall\{\xi_i\}_{i=0}^{n-1}$ => $|\sigma(\tau_i, \{\xi_i\})-I| < \epsilon$ 

$lim_{d->0}\sigma(\tau_i, \{\xi_i\}) = I \in R$

$\int \limits_{a}^{b}f(x)dx = I$
\end{document}
