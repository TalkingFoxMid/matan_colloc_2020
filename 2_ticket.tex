\documentclass[a4paper,14pt]{article}
 
%Russian-specific packages
%--------------------------------------
\usepackage{cmap}					
\usepackage[english,russian]{babel}
%--------------------------------------
 
%Hyphenation rules
%--------------------------------------
\usepackage{hyphenat}
%--------------------------------------

%Math
%--------------------------------------
\usepackage{mathtools}
\usepackage{mathrsfs}
\usepackage{amsmath}
\usepackage{amsfonts}
\usepackage{amsmath,amsfonts,amssymb,amsthm,mathtools}
%-------------------------------------- 

%Hyperlinks
%--------------------------------------
\usepackage[svgnames]{xcolor}
\usepackage[colorlinks=true, linkcolor=Blue, urlcolor=Blue]{hyperref}
%--------------------------------------
\usepackage[14pt]{extsizes}

\linespread{1.15}

\pagestyle{empty}

\begin{document}

    \begin{center}
        \textbf{2) Необходимое условие интегрируемости функции по Риману.}
    \end{center}

    \textbf{Теорема (необходимое условие интегрируемости).}
    
    Если $f$ - интегрируема на $[a, b]$, то $f$ - ограничена на $[a, b]$.

    $\;$

    \textit{Доказательство.} О/п: Пусть $f$ - интегрируема и не ограничена на $[a, b]$.

    $\;$

    \begin{center}
        {Распишем по Коши:}
    \end{center}

    $$\forall\epsilon>0\;\;\exists\delta(\epsilon)>0,\;\forall\tau,d<\delta(\epsilon)\;\;\forall\lbrace\xi\rbrace_{i=0}^{n-1}$$

    $$\vert\sigma(\tau,\lbrace\xi\rbrace_i)-I\vert<\epsilon$$

    \begin{center}
        {Возьмём $\epsilon=1\;\;\exists\delta(1)>0\;\;\forall\tau:d<\delta(1)\;\;\forall\lbrace\xi_i\rbrace$, подставим в неравенство и ограничим слева:}
    \end{center}

    $$\;\;\vert\sigma(\tau,\lbrace\xi_i\rbrace)\vert-\vert{I}\vert\leq\vert\sigma(\tau,\lbrace\xi_i\rbrace)-I\vert<1 \label{eq:1}$$

    \begin{center}
        {Т.к. $f$ - не ограничена => при разбиении на одном из отрезков она не ограничена.}
    \end{center}

    \begin{center}
        {Распишем интегральную сумму:}
    \end{center}

    $$\sigma(\tau,\lbrace{\xi_i}\rbrace)=f(\xi_0)\cdot\triangle{x_0}+\underbrace{\overset{n-1}{\underset{i=1}{\sum}}f(\xi_i)\cdot\triangle{x_i}}_{\sigma_1}$$

    \begin{center}
        {Подставим в последнее неравенство и снова ограничим слева меньшим значением:}
    \end{center}

    $$\vert{f(\xi_0)}\vert\cdot\triangle{x_0}-\vert{\sigma_1}\vert-\vert{I}\vert\leq\vert{f(\xi_0)\cdot\triangle{x_0}+\sigma_1}\vert-\vert{I}\vert<1$$

    \begin{center}
        {Перенесём в правую часть и поделим на $\triangle{x_0}$:}
    \end{center}
    
    $$\vert{f(\xi_0)}\vert<\frac{\vert\sigma_1\vert+\vert{I}\vert+1}{\triangle{x_0}}$$

    $\forall\xi_n\in[a_1, b_1]$ => ограничена - противоречие.

    \begin{center}
        {$\heartsuit\;\;$Условие \textit{необходимое}, но \textit{не достатоное}.$\;\;\heartsuit$}
    \end{center}

\end{document}
